% \usepackage[left=2cm,right=2cm,top=2cm,bottom=2cm]{geometry} %set in config.py
\usepackage[multidot]{grffile} % handles larger range of file names
\usepackage{colortbl}
% do not use natbib has it has conflicts with sphinx:
%\usepackage{natbib}
\bibliographystyle{plainnat}
%\usepackage[usenames,dvipsnames]{xcolor} % xcolor should be loaded with sphinx.sty
\usepackage{array} % for defining a new column type, if needed
\usepackage{varwidth} %for the varwidth minipage environment
\usepackage{tikz} % handles images (especially background ones, but has more capabilities if needed)
\usetikzlibrary{calc}  % seems to be used in \PlaceText command (see below) for the title text blocs
\usepackage{xspace} % for newcommand and trailing spaces
\usepackage[justification=centering]{caption}

\definecolor{gfzblue}{RGB}{0,88,156} % http://media-intranet.gfz-potsdam.de/gfz/wv/%C3%96A/Corporate%20Design%20-%20de%20PDF.pdf

% make command that prints stuff if a given command is not empty. E.g. given \rstDoi, and we want to
% print ``DOI: \rstDoi'' only if \rstDoi is not empty, then:
% \ifnotempty{\rstDoi}{DOI: \rstDoi}
\newcommand{\ifnotempty}[2]{%
    	\if\relax#1\relax%
% 			EMPTY
		\else%
  			#2%
		\fi%
}


% Sphinx styles (defined in sphinx.sty, added by sphinx in the late folder)
% Note that from version 1.4.+, to avoid conflict sphinx renamed each command, e.g. \tablecontinued
% into \sphinxtablecontinued. The former is NOT defined if in conf.py we provide
% keep_old_macro_names = False.
% Problem is, in version 1.5.2 there is still code using old macro names, thus to avoid bugs
% we still need keep_old_macro_names=True, and here we must re-define both commands for safety
%
% (side-note: we do not upgrade to sphinx newer version because they have a completely
% different templating system for latex, which looks good but, as of sphinx 1.6+ is still UNDER DEVELOPMENT.
% See e.g. http://www.sphinx-doc.org/en/1.6.3/latex.html)
%
% sphinxtablecontinued is used for longtables and prints basically ``Continued on next page''
% in sans serif. We want it normal font, but small. Thus:
\renewcommand{\sphinxtablecontinued}[1]{\small #1}
\renewcommand{\tablecontinued}[1]{\small #1}
% make table headers ``normal'' (by default is \textsf in sphinx.sty). In this case it seems there
% is no counterpart \stylethead (AARGHHH Sphinx)
\renewcommand{\sphinxstylethead}{}


% setup footers and headers in the "normal" document pages
% IMPORTANT NOTE: fancyhdr is required if the document class is NOT memoir, (which is set in config.py
% and is always true for this report)
% Note: fancyhdr is loaded in sphinx.sty % (which is inside the sphinx package, and will be copied to the latex output directory when building
% with -b latex)
\pagestyle{fancy}
% define headers and footer globally valid for everything:
\renewcommand{\headrulewidth}{0pt}
\renewcommand{\footrulewidth}{0pt}
\lhead{} % no left header
\rhead{} % no right header
\chead{} % no center header
\lfoot{\scriptsize\ifnotempty{\rstStrNum}{\rstStrNum.\ }GFZ German Research Centre for Geosciences. DOI: \rstDoi}
\cfoot{} % no center foot
\rfoot{\footnotesize \thepage} % print the page

% Make title. Override sphinx title defined in sphinxhowto.cls
% (which is inside the sphinx package, and will be copied to the latex output directory when building
% with -b latex)
% document to latex with sphinx)
% make title is overridden in sphinxhowto.cls (copied from original sphinx sphinxhowto)

% The gfz title has several blocs: authors, title bloc (title+subtitle+sub-subtitle), footers
% (scientific technical report bla bla bla), and each bloc has to be placed at a certain height.
% Separating the blocs with new lines does not preserve, obviously, the bloc placement
% after a bloc with an unexpectedly long text. In such a case, in word, we simply add/remove newlines,
% here we can't. 
% So place bloc texts inside boxes, and each box is at a specified position. Text might overlap, it will
% rarely overflow. And if it does (either overlap or overflow) then it's a signal that texts should be shorter

% First command from the web: THIS COMMAND CREATES A BOX OF TEXT which can be put ANYWHERE in the document.
% Just provide two argument x and y (left and top), plus the third argument (the text to be displayed)
\newcommand\PlaceText[3]{%
\begin{tikzpicture}[remember picture,overlay]
\node[outer sep=0pt,inner sep=0pt,anchor=north west, text width=\paperwidth-\oddsidemargin-1in-#1] 
  at ([xshift=#1,yshift=-#2]current page.north west) {#3};
\end{tikzpicture}%
}

% reset the maketitle command:
\renewcommand{\maketitle}{
	% 1) copied from sphinxhowto.cls (the original one)
	\ifsphinxpdfoutput
		\begingroup
		% These \defs are required to deal with multi-line authors; it
		% changes \\ to ', ' (comma-space), making it pass muster for
		% generating document info in the PDF file.
		\def\\{, }
		\def\and{and }
		\pdfinfo{
		  /Author (\@author)
		  /Title (\@title)
		}
		\endgroup
	\fi

	\begingroup % begin a group NOW and end it at the end of the title. So that we can set local settings like, e.g.: 
	\sffamily
	% or if we want to change font totally:
	% \fontfamily{qag}\selectfont
	
	% 2) title page
  	% custom make title:
  	% print background picture:
  	\begin{tikzpicture}[remember picture,overlay]
		\node[opacity=1,inner sep=0pt] at (current page.center){\includegraphics[width=\paperwidth]{Cover_STR-Data_en_DH.front.pdf}};
	\end{tikzpicture}%

	% make title text
	% RECALL:
	% 1st arg: the LEFT of the text box, 2nd argument the TOP of the textbox
	%author:
	\PlaceText{5.8cm}{9cm}{\large \@author \par}
	% title+substitles
	% in principle, when changing the font like {\LARGE ..} the baselineskip when issuing \\ or \newlines should
	% still be the baseline skip of the normal text. To issue a baselineskip which is the same
	% as the font size, we should enclose each change of font inside {}, and putting a \par at the end
	% e.g.: {\LARGE \bfseries \color{gfzblue} \@title \\ \par}. The solution below seems to achive the same
	% with \\[\baselineskip], but we should FIX: is then worth to adda \par at the end? For info google or see:
	% https://texblog.org/2012/08/29/changing-the-font-size-in-latex/ 
	\PlaceText{5.8cm}{13cm}{{\LARGE \bfseries \color{gfzblue} \@title \newline \par}
							\ifnotempty{\rstSubtitle}{\Large \bfseries \color{gfzblue} \rstSubtitle \newline \par}
							\ifnotempty{\rstSubSubtitle}{\large \bfseries \color{gfzblue} \rstSubSubtitle \newline \par}}
	 %title footer:
	\PlaceText{5.8cm}{22cm}{\ifnotempty{\rstStrText}{\Large \rstStrText\ \rstStrNum \newline \par}
							\ifnotempty{\rstVersion}{Version: \rstVersion} \ifnotempty{\rstDate}{\rstDate}}
	
	% FIXME: what are these? (uncomment all but setcounter, which makes sense)
  	% \py@authoraddress \par
  	% \@thanks
 
  	\setcounter{footnote}{0}
  	% \let\thanks\relax\let\maketitle\relax
	\clearpage
  
	% 3) second title page, with citations and stuff in the lower right corner:	
	% in principle, inside {\footenotesize ..} the baselineskip when issuing \\ or \newlines should
	% still be the baseline skip of the normal text, whihc is fine. For info google or see:
	% https://texblog.org/2012/08/29/changing-the-font-size-in-latex/  
	{\footnotesize
		
		\rstCitations{\ }  % add empty string at the end so that \vfill maybe works when rstCitations is empty

		% lower right corner:
		\begin{flushright}\vfill
			\begin{tabular}{r}
				{\normalsize Imprint} \\
				\\
				\includegraphics{gfz_wordmark_blue.png} \\
				\\
				Telegrafenberg \\
				D-14473 Potsdam  \\ 
				\\
				Published in Potsdam, Germany \\
				\rstPublicationMonth\ \rstPublicationYear \\
				\\
				\ifnotempty{\rstIssn}{ISSN \rstIssn}\\
				\\
				DOI: \rstDoi \\
				\ifnotempty{\rstUrn}{URN: \rstUrn}\\
				\\
				This work is published in the GFZ series \\
				Scientific Technical Report (STR) \\
				and electronically available at GFZ website \\
				\href{http://www.gfz-potsdam.de}{{\color{black}{www.gfz-potsdam.de}}} \\
				\\
				\href{http://creativecommons.org/licenses/by/4.0/}{\includegraphics{creativecommon_80x15.png}}
			\end{tabular}
		\end{flushright}
	}
	% no header and footer, and then clear page
	\thispagestyle{empty}
	\clearpage

	% make title 3rd page:
	% RECALL:
	% 1st arg: the LEFT of the text box, 2nd argument the TOP of the textbox
	% see notes in title+substitles above about \par at the end
	%author:
	\PlaceText{4cm}{6.5cm}{\large \rstAuthorsWithAffiliations \par}
	 %title+substitles
	\PlaceText{4cm}{10cm}{{\LARGE \bfseries \@title \newline \par}
						  \ifnotempty{\rstSubtitle}{\Large \bfseries \rstSubtitle \newline \par}
						  \ifnotempty{\rstSubSubtitle}{\large \bfseries \rstSubSubtitle \newline \newline \par}
						  \ifnotempty{\rstAffiliations}{\normalsize \mdseries \rstAffiliations\par}
						  }
	 %title footer:
	\PlaceText{4cm}{23.5cm}{\ifnotempty{\rstStrText}{\Large \rstStrText\ \rstStrNum \newline \par}
							\ifnotempty{\rstVersion}{Version: \rstVersion} \ifnotempty{\rstDate}{\rstDate}}

	\thispagestyle{empty}
	\clearpage
	
	\endgroup
}
