\usepackage[left=2cm,right=2cm,top=2cm,bottom=2cm]{geometry} % see geometry.pdf on how to lay out the page. There's lots.
\geometry{a4paper} % or letter or a5paper or ... etc
\usepackage[multidot]{grffile}
\usepackage{colortbl}
\usepackage{fancyhdr}
\usepackage{natbib}
\bibliographystyle{plainnat}
\usepackage[usenames,dvipsnames]{xcolor}
\usepackage{array} % for defining a new column type, if needed
\usepackage{varwidth} %for the varwidth minipage environment
\usepackage{tikz}
\usepackage{xspace} % for newcommand and trailing spaces

% \usepackage[colorlinks=true,linkcolor=blue,urlcolor=NavyBlue,citecolor=Fuchsia]{hyperref}
% \urlstyle{same}
\definecolor{gfzblue}{RGB}{7,90,154} 

% create the command to print the picture background:
% arguments are the x and y coordinates (the latter negtive to shift up)
\newcommand{\printBgPic}[2]{%
	\makebox[0pt][l]{%
		\raisebox{#2}[0pt][0pt]{%
		\hbox{\hspace{#1}\includegraphics[width=3\textwidth]{gfz-bg.pdf}}%
    	}%
	}%
}

% create the command to print the logos in the title page:
\newcommand{\printbgpiccorners}{%
\makebox[0pt][l]{% make it in a box to behave like a html float (out of doc flow)
	% upper right picture
	\begin{tikzpicture}[remember picture, overlay]
		\node [anchor=north east, inner xsep=2cm, inner ysep=1.2in + \voffset]  at (current page.north
		east) {\includegraphics{GFZ-Logo_eng_RGB.png}};
	\end{tikzpicture}%
	% upper left
	\begin{tikzpicture}[remember picture, overlay]
		\node [anchor=north west, inner xsep=2cm, inner ysep=0.6in + \voffset]  at (current page.north
		west) {\includegraphics{gfz_wordmark_blue.png}};
	\end{tikzpicture}%
	% lower left
	\begin{tikzpicture}[remember picture, overlay]
		\node [anchor=south west, inner xsep=2cm, inner ysep=1in + \voffset]  at (current page.south west)
			{www.gfz-potsdam.de};
	\end{tikzpicture}%
	% lower right:
	\begin{tikzpicture}[remember picture, overlay]
		\node [anchor=south east, inner xsep=2cm, inner ysep=1in + \voffset]  at (current page.south east)
			{\includegraphics{helmholtz_logo_blue.png}};
	\end{tikzpicture}%
}%
}

% setup footers and headers in the "normal" document pages
\pagestyle{fancy}
% define headers and footer globally valid for everything:
\renewcommand{\headrulewidth}{0pt}
\renewcommand{\footrulewidth}{0pt}
\lhead{} % no left header
\rhead{} % no right header
\chead{} % no center header
\lfoot{\scriptsize \rstStrNum. GFZ German Research Centre for Geosciences. DOI: \href{\rstDoiUrl}{\rstDoiStr}}
\cfoot{} % no center foot
\rfoot{\footnotesize \thepage} % print the page

% maketitle macro (subdivided in several steps):

% start with makeatletter cause we will use the @ as keyword in maketitke
\makeatletter
% create the command that makes the title string:
% arguments width, y. E.g. \maketitlestring{.8\linewidth}{20em}
% Note: the title box right side will be fixed to the right page side, so the
% title will be right aligned if width < \textwidth
\newcommand{\maketitletext}[2]{
	% leave the space below!! (problems ith vspace?) otherwise 

	\vspace*{#2}%
	\hfill\begin{tabular}{@{}p{#1}@{}}
		%\multicolumn{1}{@{}c@{}}{Some title} \\  % this is if we wanted to center the title inside the tabular (we don't)
		{\large \par \@author \par} \\
		{\LARGE \sffamily \bfseries \color{gfzblue} \@title \par}\\
	  	{\Large \sffamily \bfseries \color{gfzblue} \rstSubtitle \par} \\
	  	{\large \sffamily \bfseries \color{gfzblue} \rstSubSubtitle \par} \\
	  	{\Large Scientific Technical Report \rstStrNum} \\
	  	{\par \normalsize GIPP Experiment- and Data Archive}
	\end{tabular}
	% add a space below for the same reason (unkwnown) we added one above (FIXME. check why)

}
\makeatother

\makeatletter
% reset the maketitle command:
\renewcommand{\maketitle}{
	% 1) copied from sphinxhowto.cls (the original one, in the annexed one is commented out to
	% override the maketitle, as we are doing)
	\ifsphinxpdfoutput
		\begingroup
		% These \defs are required to deal with multi-line authors; it
		% changes \\ to ', ' (comma-space), making it pass muster for
		% generating document info in the PDF file.
		\def\\{, }
		\def\and{and }
		\pdfinfo{
		  /Author (\@author)
		  /Title (\@title)
		}
		\endgroup
	\fi

  	% 2) title page
  	% custom make title:
  	% print background picture:
	\printBgPic{-1.5\textwidth}{-1.08\totalheight}
  	% print corner pictures
	\printbgpiccorners
	% make title text (preformatted, choose width and y of the title box as
	% arguments):
	\maketitletext{.8\linewidth}{20em}
	% FIXME: what are these? (uncomment all but setcounter, which makes sense)
  	% \py@authoraddress \par
  	% \@thanks
 
  	\setcounter{footnote}{0}
  	% \let\thanks\relax\let\maketitle\relax
	\clearpage
  
	% 3) second title page, with citations and stuff in the lower right corner:	  
	{\footnotesize
		
		Recommended citation (mandatory): 							\newline\newline
		\@author\ (\rstPublicationYear): \@title. DOI: \href{\rstDoiUrl}{\rstDoiStr} 		\newline\newline\newline
		Supplementary datasets (if any): 							\newline\newline
		\rstSupplDdatasets 											\newline\newline\newline
		Recommended citation for chapter (if any): 					\newline\newline
		\rstCitationChapter											\newline\newline\newline
		The report and the datasets are supplements to: (if any):	\newline\newline
		\rstSupplementsTo
		
		% lower right corner:
		\begin{flushright}\vfill
			\begin{tabular}{r}
				{\normalsize Imprint} \\
				\\
				\includegraphics{gfzlogo_ur.pdf} \\
				\\
				Telegrafenberg \\
				D-14473 Potsdam  \\ 
				\\
				Published in Potsdam, Germany \\
				\rstPublicationMonth\ \rstPublicationYear \\
				\rstIssn \\
				DOI: \href{\rstDoiUrl}{\rstDoiStr} \\
				URN: \rstUrn \\
				\\
				This work is published in the GFZ series \\
				Scientific Technical Report (STR) \\
				and electronically available at GFZ website \\
				www.gfz-potsdam.de \\
				\\
				\href{http://creativecommons.org/licenses/by-sa/4.0/}{\includegraphics{creativecommon_80x15.png}}
			\end{tabular}
		\end{flushright}
	}
	% no header and footer, and then clear page
	\thispagestyle{empty}
	\clearpage

	% 4) third page title, the same as the first but no background
	% (and a little bit left-shifted)
	\maketitletext{.85\linewidth}{20em}
	\thispagestyle{empty}
	\clearpage
}
\makeatother